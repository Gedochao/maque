\documentclass[12pt]{wzmgr}
\special{papersize=210mm,297mm}
% Wersja XeLaTeX
\usepackage{mathtools}
\usepackage{CJK}
\usepackage{xeCJKfntef}
\usepackage[polish]{babel}
\usepackage[no-math]{fontspec}
\usepackage{xeCJK,makeidx}
\usepackage{setspace,etoolbox,csquotes,indentfirst}
\usepackage[top=2.5cm,left=2.5cm,right=2.5cm,bottom=2.5cm]{geometry}
\usepackage{url}
\usepackage[usenames,dvipsnames,svgnames,table]{xcolor}
\usepackage{blindtext}
\usepackage[inline]{enumitem}
\usepackage{xcolor}
\makeatletter
\g@addto@macro{\UrlBreaks}{\UrlOrds}
\makeatother
\setmainfont[Mapping=tex-text]{Adobe Text Pro}
\setsansfont[Scale=0.88]{IPAexGothic}
%\setCJKmainfont{AR PL UMing CN}
%\setCJKsansfont{SimHei}
%\setCJKmonofont{SimHei}
%\setCJKmainfont[BoldFont=FandolSong-Bold.otf,ItalicFont=FandolKai-Regular.otf]{FandolSong-Regular.otf}
\setCJKmainfont[BoldFont={SimHei},ItalicFont={SimHei}]{SimSun}
% \setmainfont[
%  BoldFont={Minion Pro Bold}, 
%  ItalicFont={Minion Pro Italic},
%  BoldItalicFont={Minion Pro Bold Italic}
%  ]{Linux Libertine O}
\setCJKsansfont[BoldFont=FandolHei-Bold.otf]{FandolHei-Regular.otf}
\setCJKmonofont{FandolFang-Regular.otf}
\newCJKfontfamily{\ipaexgothic}{IPAexGothic}
\newCJKfontfamily{\korm}{NanumMyeongjo}
\newtoggle{brudnopis}\newtoggle{prowincjepy}
\togglefalse{brudnopis} % change \toggletrue to \togglefalse to disable comments
\toggletrue{prowincjepy}
\newcommand{\prowincja}[1]{\iftoggle{prowincjepy}{ (#1)}{}}
\newcommand{\Korean}{\korm\CJKspace}
\newcommand{\toponim}[1]{\textsf{#1}}
\newcommand{\nazwisko}[1]{#1}
\newcommand{\pinyin}[1]{\textit{#1}}
\newcommand{\fnm}{\footnotemark}
\newcommand{\komentarz}[1]{\iftoggle{brudnopis}{\colorbox{Red}{#1}}}
\newcommand{\boldzh}[1]{\CJKunderdot{\textbf{#1}}}
\author{Ged}
\title{Gramatyka opisowa - opracowanie do egzaminu}
\nrwersji{0.0}
\miejsce{Poznań}
%\hyphenation{ La\.n-kā-va-tā-ra}
\makeindex
\begin{document}
\onehalfspacing
\maketitle
\chapter{Spójniki}
%%%%%%%%%%%%%%%%%%%%%
% TEORIA
%%%%%%%%%%%%%%%%%%%%%
\section{Skrót teorii}
Mogą występować jako słowa puste, a mogą być równocześnie przysłówkami lub
przyimkami. Przyimki które mogą być spójnikami: 跟、和、同、与

Spójniki, na które zwrócono uwagę na wykładzie:
\begin{itemize}
\item 和
\item 及,以及 -- działają podobnie do 和, łączą rzeczowniki/frazy rzeczownikowe; 以及
  może też łączyć frazy czasownikowe i zdania składowe
\item 或者,或 -- spójnik alternatywy, może łączyć wyrazy, frazy, zdania składowe;
nie mylić z 还是, które występuje w zdaniach pytajnych 
\item 与其,宁可,不如,宁愿,宁肯
	\begin{itemize}
		\item 与其(słabsza opcja)宁可(preferowana opcja)
		\item 不如,宁愿,宁肯 mogą wystąpić zamiast 宁可
	\end{itemize}
\item 而 -- współrzędne, przeciwstawne lub następujące po sobie
\item 并,并且 -- łączy zdania składowe i frazy czasownikowe; 并且 łączy całe zdania
\item 不但……而且; 不但……并且 - nie tylko (…), ale wręcz (…)
\item 反而
\item 况且,何况,再说 -- uzupełnienie informacji z 1szego zdania składowego; 何况
występuje w zdaniach retorycznych 因为,由于 - jeśli używamy 由于, to musi ono wystąpić w 1szym zdaniu składowym 
\item 所以,因此,因而
\item 既然,既 -- "ze względu na to, że (…), to (…)" 
	\begin{itemize}
	  \item często występuje z 就、便、也;
	\end{itemize}
\item 虽然,尽管,但是,可是,然而
\item 即使 -- chociaż, mimo że
\item 只有,只要
	\begin{itemize}
		\item 只有 warunek,才 tylko wtedy można
		\item 只要 warunek,就 dopiero wtedy można
	\end{itemize}
\item 无论,不论,不管
	\begin{itemize}
		\item 不管你A不A -- niezależnie czy A czy nie A, to (…)
		\item 不管A还是A -- niezależnie czy A czy A, to (…)
		\item 不管A -- niezależnie 谁,什么, etc, to (…)
	\end{itemize}
\item 除非 -- tylko wtedy gdy; może też oznaczać "za wyjątkiem (tego co stoi po
除非)
\item 以便 -- po to, by
\item 以免,免得,省得 -- aby uniknąć
\end{itemize}
%%%%%%%%%%%%%%%%%%%%%
% ĆWICZENIA
%%%%%%%%%%%%%%%%%%%%%
\section{Ćwiczenia}
刘月华等 著-Practical Modern Chinese Grammar-商务印书馆
(2004) - dalej PMCG
\subsection{Ćwiczenie 1 - PMCG}

strona 350 (371 w pdf)
 


Wskazać spójniki i wyjaśnić, na czym polega
ich rola.
\begin{enumerate} 
\item 今天下午,老王把复明\boldzh{和}我找去谈话了。%1

和 //można połączyć wyrazy i frazy, głównie rzeczownikowe (nominalne);
zwykle nie może łączyć fraz werbalnych, acz są wyjątki. Zdań nie można łączyć 和.
\item 这一切对我是多么热情的支持\boldzh{和}鼓励啊!%2

和 łączy 2 wyrazy
\item 你今天来\boldzh{或者}明天来都可以。%3

或者 łączy frazy werbalne, zdania
\item 寄往城里的信用两毛钱的邮票\boldzh{还是}四毛的? %4

还是 pytanie alternatywne
\item 这是一个美丽\boldzh{而}动人的神话故事。%5

而
\item 对什么问题都应该想得深一些\boldzh{和}远一些。%6

和 tutaj łączy frazy przymiotnikowe
\item 遇到下雨,多云\boldzh{或者}有雾的天气,他们也坚持出工。%7

或者 różnego rodzaju frazy
\item \boldzh{虽然}很忙,\boldzh{可是}我们都感到我们的生活是愉快的,幸福的。%8

可是(wchodzi w konstrukcję ze 虽然) łączy zdania
\item \boldzh{既然}是试验,\boldzh{就}别怕失败。%9

既然(z 就)
\item 今天,\boldzh{不但}在生产上,国防上实现了自动化或者半自动化,人们的日常生活\boldzh{也}开始进入自动化的时代。%10

不但 (z 也) łączy zdania
\item 他\boldzh{尽管}还在病休,\boldzh{可是}还抓紧时间刻苦自学外语。%11

尽管,可是 łączy zdania
\item 这个任务咱们\boldzh{不但}要接受下来,\boldzh{而且}还要完成得快,完成得好。%12

不但, 而且 łączy zdania
\item 老王\boldzh{就是}听了和自己不同的意见,\boldzh{也}从不发火。%13

就是, 也
\item \boldzh{倘若}能倒退回十年,让我重过学习生活,该多好啊!%14

倘若
\item \boldzh{不论}做什么工作,\boldzh{都}不能粗心大意。%15

不论, 都
\item \boldzh{宁可}自己苦一点儿,\boldzh{也}要帮助别人解除困难。%16 

宁可,也 konstrukcja łącząca zdania
\item \boldzh{与其}跪着生,\boldzh{不如}站着死。 %17

与其,不如 konstrukcja łącząca zdania
\item \boldzh{只有}艰苦奋斗,\boldzh{才}能获得成功 %18

只有,才 konstrukcja łącząca zdania
\item 你\boldzh{只要}肯下苦功夫,\boldzh{就}能学好汉语。 %19

只要, 就 konstrukcja łącząca zdania
\item 我们应该继承\boldzh{并}发扬中华民族的光荣传统。 %20

并
\end{enumerate}

\subsection{Ćwiczenie 2 - PMCG}

strona 352 (373 w pdf)

Wstawić odpowiednie spójniki spośród wymienionych.

(一) 只要; 无论(不论,不管); 因为; 所以; 尽管; 但是; 既然; 即使 (就是,哪怕); 以免; 固然; 虽然; 可是

\begin{enumerate} 

\item \boldzh{既然}你不愿意支持我的事业,那么咱俩就分道扬镳吧,好离好散。%1
\item \boldzh{不管}天气多么冷,他都坚持洗冷水澡。%2
\item 我们\boldzh{只要}努力学习,就能获得好成绩。%3 

łączy się z 就; 只要 wyraża warunek wystarczający
\item 小王\boldzh{因为}没有按照规定的方法进行生产,\boldzh{所以}除了事故。 %4

związek przyczynowo-skutkowy
\item 妹妹替大哥着急,\boldzh{可是}又没有能力帮助他。 %5

wyraża zwrot, jest jakieś "ale"
\item 老人\boldzh{虽然}身体不很好,每天还坚持工作七八个小时。 %6
\item 今晚\boldzh{不管}多晚,你一定要到我家来一趟。 %7
\item A:我跟你的看法不同,也很难一致起来。 %8

B:\boldzh{既然}咱们俩的看法不一致,那就无法在一起合作了。
\item 练习的题目\boldzh{虽然}难了一些,\boldzh{但是}[lub:\boldzh{固然}]难有难的好处。%9
\item 自己主观上不努力,客观条件\boldzh{即使}在好,也不起作用。 %10

nawet jeśli
\item 这个任务很急,\boldzh{即使}几天几夜不睡觉,也要按时把它完成。%11
\item 关于办公司的事,你有什么意见就说吧,\boldzh{即使}你想全盘否定我的意见,我也不会生气的。 %12
\item 新媳妇\boldzh{只要}一下班回到家,\boldzh{就}闲不住地帮助婆婆干这干那。 %13
\item 这里有高压电线,请不要靠近,\boldzh{以免}发生危险。 %14
\end{enumerate}

(二)还是;不但……而且……;或者;宁可;不但……反而……
\begin{enumerate}
\item 由于他错签了一个合同,这个月他\boldzh{不但}没有受到表扬和奖励,\boldzh{反而}还被扣罚了本月的奖金。 %1
\item 老李今天晚上来\boldzh{还是}明天一早来,他也没说定。 %2

wyraża niepewność
\item 我们\boldzh{不但}要学习好,\boldzh{而且}还要品德好,身体好。 %3
\item 你用中文讲\boldzh{或者}用英文讲都可以,我们都会英语。 %4
\item 我们的任务\boldzh{不但}按期完成了,\boldzh{而且}完成得很出色。 %5
\item 为了我的事业,我\boldzh{宁可}晚一点儿结婚,\boldzh{或者}不结婚。 %6

宁可 wyraża preferencję jakiejś opcji
\end{enumerate}
\chapter{Partykuły}
\chapter{Onomatopeje i wykrzykniki}
\chapter{Zdanie i jego części}
\chapter{Przydawka i okolicznik}
\chapter{Komplementy}
\chapter{Zdanie proste}
\printindex
\end{document}