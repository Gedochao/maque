\chapter*{Streszczenie}
\addcontentsline{toc}{chapter}{Streszczenia}
Madżong to gra dla czterech graczy pochodząca z Chin. Pojawiła się po raz
pierwszy w Chinach pod koniec XIX wieku, będąc kombinacją wielu istniejących
wcześniej gier karcianych, jak \pinyin{shihu} i \pinyin{madiao}. 

Niniejsza praca opisuje historię madżonga, jego wpływ na kulturę Chin i świata,
a także prezentuje i porównuje najważniejsze spośród wielu wariantów jego zasad.

W pierwszym rozdziale opisano historię madżonga. Pierwsze zapisy o
istnieniu tej gry pochodzą z lat siedemdziesiątych XIX wieku, jednakże popularna
stała się dopiero w latach dwudziestych XX wieku. W 1966 roku zakazano gry w
madżonga w Chinach. Zakaz zniesiono w 1998 roku, gdy madżong został uznany za
dyscyplinę sportową.

W drugim rozdziale przedstawiono szczegółowo międzynarodowe zasady turniejowe,
które są używane na zawodach sportowych.

W trzecim rozdziale omówiono inne spośród najważniejszych wariantów zasad gry w
madżonga: zasady klasyczne, kantońskie, japońskie i amerykańskie. Wariant
japoński \romaji{rīchi} jest najbardziej skomplikowany, podczas gdy zasady
amerykańskie są uproszczone w porównaniu do innych. Nie da się jednoznacznie
stwierdzić, czy wariant klasyczny, czy kantoński w starym stylu jest starszym.
Najprawdopodobniej przed nimi istniały inne zasady.


