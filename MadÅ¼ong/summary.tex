\summary
Choć rozpowszechniony jest pogląd o pochodzeniu madżonga z czasów Konfucjusza, w
rzeczywistości gra spełniająca przyjętą na potrzeby niniejszej pracy definicję
pojawiła się po raz pierwszy pod koniec XIX wieku. Nie był to w pełni nowy
twór, lecz raczej kombinacja zasad wielu istniejących wcześniej gier karcianych,
jak \pinyin{shihu} czy \pinyin{madiao}, połączona ze stopniową wymianą
papierowych kart na trwalsze kamienie.

Gra stała się naprawdę popularna dopiero w latach dwudziestych XX wieku, kiedy
zetknęli się z nią w Szanghaju, mającym wówczas status specjalnej strefy
względem przepływu obcokrajowców, biznesmeni z Ameryki Północnej i Europy. Gra
uległa standaryzacji i zaczęła być eksportowana poza Chiny. Reguły gry z tego
czasu później zaczęły być nazywane zasadami chińskimi klasycznymi. 

Równolegle w Japonii i Stanach Zjednoczonych rozwijały się tamtejsze warianty
zasad gry. W latach pięćdziesiątych XX wieku wytworzyły się reguły
odmiany japońskiej \romaji{rīchi}, natomiast w 1937 roku powstał standard
madżonga amerykańskiego. Podczas gdy w Japonii zasady gry były coraz bardziej
skomplikowane, w Ameryce stawały się one coraz prostsze.

Jako że madżong pierwotnie był grą hazardową, gra w niego na terenie Chin stała
się nielegalna w 1966 roku, gdy zakazano hazardu w całej Chińskiej Republice
Ludowej. Przyczyniło się to do rozwoju zasad hongkońskich, jako że w Hong Kongu
zakaz ten nie obowiązywał. W roku 1998 opracowano międzynarodowe zasady turniejowe,
które były pozbawione czynnika hazardowego i przyczyniły się do odrodzenia gry w
Chinach. Madżong jest też stałym elementem kultury popularnej. Występuje w
licznych książkach, filmach i innych mediach. Od 2002 roku organizowane są także
międzynarodowe zawody sportowe w grze w madżonga.

Na świecie istnieją ponad 42 warianty zasad gry, w większości charakterystyczne
dla danego obszaru świata. Zazwyczaj użytkownicy każdego z nich nie są świadomi
istnienia pozostałych, choć różnice między nimi często są dosyć znaczące.

Międzynarodowe zasady turniejowe zawierają w sobie elementy wielu powstałych
wcześniej na terenie Chin odmian madżonga, lecz pozbawione elementu hazardowego
zwiększyły wpływ indywidualnych umiejętności graczy. Tym samym są one jednym z 2
wariantów najczęściej używanym na zawodach sportowych. Drugim jest madżong
japoński \romaji{rīchi}, który choć w dalszym ciągu posiada wiele czynników
hazardowych, ze względu na swoją złożoność i wyważenie różnych elementów gry,
również wymaga w zdecydowanie większym stopniu umiejętności niż szczęścia od
gracza. 

Choć wszystkie analizowane w niniejszej pracy warianty zasad mają wiele
elementów wspólnych, jak podstawowe mechanizmy przebiegu gry (kolejność tur,
wymagania dotyczące wygrywającej ręki, proces budowy muru, dobierania i
odrzucania kamieni), znacząco różnią się między sobą sposobem liczenia punktów
oraz układami punktowanymi, a niekiedy również używanym zestawem kamieni i
pomniejszymi zasadami, jak obecność martwego muru czy charakterystyczna dla
wariantu \romaji{rīchi} \romaji{dora}.

Nie jest do końca jasnym, który ze starych wariantów (zasady chińskie klasyczne
czy kantońskie w starym stylu) powstał wcześniej, jednakże najprawdopodobniej
oba wywodzą się z nieustandaryzowanych, różnorodnych zasad, zgodnie z którymi
grano na przełomie XIX i XX wieku.

Można dostrzec najwięcej podobieństw pomiędzy wariantami pochodzącymi z bliskich
czasowo epok. Międzynarodowe zasady turniejowe, \romaji{rīchi} oraz zasady
hongkońskie w nowym stylu wszystkie przestrzegają rygorystycznych wymagań
dotyczących układania odrzuconych kamieni oraz oznaczania pochodzenia
tych dobranych od innych graczy. Zasady klasyczne oraz kantońskie w starym
stylu, które pochodzą z wczesnego okresu rozwoju gry, nie narzucają podobnych
oczekiwań wobec uczestników.
