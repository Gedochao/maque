\chapter{Zasady gry}
\section{Różnorodność zasad gry w madżonga w Chinach i na świecie}
\epigraph{麻将源于中国属于世界 \\ 
\footnotesize \pinyin{Májiàng yuán yú Zhōngguó shǔyú shìjiè} \normalsize \\
Mahjong pochodzi z Chin, lecz
należy do świata.}{Yu Guangyuan\footnote{Yu Guangyuan (于光远 Yú
Guāngyuǎn) (1915-2013) -- znany chiński filozof i działacz Komunistycznej Partii
Chin; prezes Światowej Organizacji madżonga w latach 2005 do 2013 (Sina
Xīnlàng Lùntán 2010; Wikipedia 2016).}
\\ (Shìjiè Májiàng Zǔzhī 2006)}

Aktualnie na świecie istnieje kilkadziesiąt rozróżnialnych zestawów zasad gry
w madżonga. W wielu przypadkach gracze pochodzący z danego obszaru nie są
nawet świadomi istnienia innych wariantów niż ten, w który grają sami. Według
klasyfikacji Toma Slopera jest ich 42 (Sloper 2002), jednakże nie uwzględnia on
wielu spośród mniej znanych lub różniących się od nich drobnymi szczegółami, jak
choćby odmiana południowoafrykańska (zbliżona do zasad pochodzących z Hongkongu)
(World Series of Mahjong 2015).

Warto także wspomnieć, że nawet współcześnie powstają nowe warianty gry, które
cieszą się zainteresowaniem wprowadzając innowacyjne zasady, jednakże przestrzegając w
wystarczającym stopniu tych istniejących wcześniej, aby wciąż klasyfikowano je
jako reguły gry w madżonga. Przykładem takich zasad jest japoński madżong
\romaji{washizu} (鷲巣), zwany również madżongiem transparentnym (麻雀透明
\romaji{mājan tōmei}). Pochodzi on z mangi pod tytułem \textit{,,Akagi:
Geniusz, który zstąpił w ciemność''} (アカギ 〜闇に降り立った天才〜 \romaji{Akagi: Yami ni
Oritatta Tensai}), która jest wciąż wydawana od 1992 roku przez Nobuyuki
Fukumoto (福本伸行 Fukumoto Nobuyuki). Esencją sukcesu \romaji{washizu} jest zmiana
trzech czwartych kamieni na transparentne, co drastycznie zmienia przebieg
rozgrywki. Pomijając tę zmianę i sposób podziału kamieni, madżong transparentny
w zasadzie nie różni się od japońskich zasad turniejowych \romaji{rīchi} (patrz:
strona \pageref{riichi}), jednakże mimo to stał się on popularny na całym
świecie. Podobnych przykładów powstających współcześnie i zyskujących rozgłos
nowych zasad gry w madżonga jest wiele (Miller 2015).

W związku z mnogością różnorodnych zasad autor nie opisze ich wszystkich w
treści niniejszej pracy, skupiając się na szczegółowym opisaniu międzynarodowych
zasad turniejowych (patrz: strona \pageref{guobiao}) oraz skróconym
scharakteryzowaniu 6 innych spośród najpopularniejszych i najbardziej wpływowych
wariantów. 
