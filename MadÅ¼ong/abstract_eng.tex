\chapter*{Abstract}

Mahjong is a four player chinese game. It has first appeared by the end
of the nineteenth century, being a combination of several older card games, like
\pinyin{shihu} and \pinyin{madiao}.

This thesis describes the history of mahjong, its impact on chinese and global
culture. It also presents and compares the most important of the variants of its
rules.

The first chapter is dedicated to the historical origins and development of
mahjong. Even though the game first came to existance by the end of the
nineteenth century, it only began to be truly popular in the twenties of the
twentieth century. In 1966 the game was banned in China, but in 1998 the ban was
cancelled and the game got recognised as a sports discipline.

The second chapter describes the international competition rules of mahjong,
which are used in mahjong championships.

The third chapter presents other variants of mahjong rules: classical,
Cantonese, Japanese and American. The Japanese rules are the most complicated,
while the American variant is very simplistic in comparison to the others. It is
difficult to tell whether the classical or Cantonese variant is the oldest.
Within all probability, other, even older rules existed before them.