% Please add the following required packages to your document preamble:
% \usepackage{multirow}
%\fan{Wielkie Cztery Wiatry}{大四喜}{Dà Sì Xǐ}{Trójki lub \pinyin{gangi} z każdego z 4 kamieni wiatrów.}
\section*{Suplement}
\tiny
\label{tab:fan2}
\begin{longtable}[]{|c|c|c|}
\caption{81 układów \pinyin{fan}} \\
%
%\hline
%\multicolumn{3}{|c|}{\centering 81 układów \pinyin{fan}}                       
% \\
\hline \multicolumn{1}{|c|}{Punkty} & \multicolumn{1}{c|}{Nazwa} & \multicolumn{1}{c|}{Opis} \\ \hline
\endfirsthead
\caption{c.d. 81 układów \pinyin{fan}} \\
%\hline
%\multicolumn{3}{|c|}{\centering 81 układów \pinyin{fan}}                       
% \\
\hline \multicolumn{1}{|c|}{Punkty} & \multicolumn{1}{c|}{Nazwa} & \multicolumn{1}{c|}{Opis} \\ \hline
\endhead
\hline
\caption{(\pinyin{Shìjiè Májiàng Zǔzhī} 2014)} 
\endfoot
\multirow{7}{*}{88}    &  \fan{Wielkie Cztery Wiatry}{大四喜}{Dà Sì Xǐ}        
					   &  Trójki lub \pinyin{gangi} z każdego z 4 kamieni wiatrów.                
					   \\ \cline{2-3} 
                       &  \fan{Wielkie Trzy Smoki}{大三元}{Dà Sān Yuán}                     
                       &  Trójki lub \pinyin{gangi} z każdego z 3 kamieni smoków.                   
                       \\ \cline{2-3} 
                       &  \fan{Wszystko Zielone}{绿一色}{Lǜ Yī Sè}                     
                       &  \tabsplit{Ręka złożona tylko i wyłącznie z ,,zielonych'' kamieni:}{2, 3, 4, 6 i 8 z talii bambusów oraz zielonych smoków.}                    
                       \\ \cline{2-3} 
                       &  \fan{Dziewięć Bram}{九莲宝灯}{Jiǔ Lián Bǎo Dēng}                     
                       &  \tabsplit{Kombinacja kamieni 1, 1, 1, 2, 3, 4, 5, 6, 7, 8, 9, 9, 9 w jednej talii}{(odchodzi od standardowej struktury ręki).}                    
                       \\ \cline{2-3} 
                       &  \fan{Cztery \pinyin{Gangi}}{四杠}{Sì Gāng}                   
                       &  Dowolna reka zawierająca 4 \pinyin{gangi}.               
                       \\ \cline{2-3} 
                       &  \fan{Siedem Kolejnych Par}{连七对}{Lián Qī Duì}                     
                       &  Siedem par kolejnych kamieni w jednej talii (odchodzi od standardowej struktury ręki).                     
                       \\ \cline{2-3} 
                       &  \fan{Trzynaście Sierot}{十三幺}{Shísān Yāo}                     
                       &  \tabsplit{Po jednym kamieniu z każdego z honorów oraz kamieni 1 i 9 z każdej z talii,}{a także jeden duplikat dowolnego z wymienionych.}                     
                       \\ \hline
\multirow{6}{*}{64}    &  \fan{Same Terminalne}{清幺九}{Qīng Yāo Jiǔ}                        
					   &  Cała ręka złożona tylko i wyłącznie z kamieni terminalnych.                 
					   \\ \cline{2-3} 
                       &  \fan{Małe Cztery Wiatry}{小四喜}{Xiǎo Sì Xǐ}                        
                       &  Ręka zawierająca 3 trójki (lub \pinyin{gangi}) z 3 wiatrów oraz parę z czwartego.                     
                       \\ \cline{2-3} 
                       &  \fan{Małe Trzy Smoki}{小三元}{Xiǎo Sān Yuán}                        
                       &  Ręka zawierająca 2 trójki lub \pinyin{gangi} z 2 smoków oraz parę z trzeciego.                     
                       \\ \cline{2-3} 
                       &  \fan{Same Honory}{字一色}{Zì Yī Sè}                        
                       &  Ręka złożona z samych honorów.                    
                       \\ \cline{2-3} 
                       &  \fan{Cztery Zakryte Trójki}{四暗刻}{Sì Ànkè}                        
                       &  Ręka zawierająca 4 zakryte trójki (lub \pinyin{gangi}).                     
                       \\ \cline{2-3} 
                       &  \fan{Czyste Terminalne Sekwensy}{一色双龙会}{Yīsè Shuānglónghuì}                        
                       &  Ręka złożona z 2  identycznych sekwensów 1, 2, 3 oraz 7, 8, 9, a także pary z 5, wszystko w jednej talii.                      
                       \\ \hline
\multirow{2}{*}{48}    &  \fan{Poczwórny Sekwens}{一色四同顺}{Yīsè Sì Tóngshùn}                        
					   &  Ręka zawierająca 4 egzemplarze identycznego sekwensu w tej samej talii.                    
					   \\ \cline{2-3} 
                       &  \fan{Kolejne Cztery\\Czyste Trójki}{一色四节高}{Yīsè Sì Jiégāo}                           
                       &  Ręka zawierająca 4 trójki lub \pinyin{gangi} 4 kolejnych kamieni w jednej talii.                     
                       \\ \hline
\multirow{3}{*}{32}    &  \fan{Kolejne Cztery\\Czyste Sekwensy}{一色四步高}{Yīsè Sì Bù Gāo}                     
					   &  \tabsplit{Ręka zawierająca 4 sekwensy w jednej talii, w których rozpoczynające je (najniższe) kamienie są 4 kolejnymi,}{odległymi od siebie o 1 lub 2.}                     
					   \\ \cline{2-3} 
                       &  \fan{Trzy \pinyin{Gangi}}{三杠}{Sān Gāng}                        
                       &  Ręka zawierajaca trzy \pinyin{gangi}.                     
                       \\ \cline{2-3} 
                       &  \fan{Same Terminalne i Honory}{混幺九}{Hùn Yāo Jiǔ}                        
                       &  Ręka składająca się wyłącznie z honorów oraz kamieni terminalnych.                     
                       \\ \hline
\multirow{9}{*}{24}    &  \fan{Siedem Par}{七对}{Qī Duì}                        
					   &  \tabsplit{Ręka składająca się z 7 różnych par}{(odchodzi od standardowej struktury ręki).}                     
					   \\ \cline{2-3} 
                       &  \fan{Większe Honory\\i Kamienie Zszywane}{七星不靠}{Qī Xīng Bù Kào}                        
                       &  \tabsplit{Ręka składająca się z pojedynczego egzemplarza każdego z 7 honorów}{oraz pojedynczych kamieni z sekwencji zszywanych (na przykład 3-6-9 lub 2-5-8) w dowolnej z talii\\(odchodzi od standardowej struktury ręki).}                     
                       \\ \cline{2-3} 
                       &  \fan{Same Parzyste Trójki}{全双刻}{Quán Shuāng Kè}                        
                       &  Ręka zawierająca 4 trójki (lub \pinyin{gangi}) z kamieni o numerach parzystych (2, 4, 6, 8).                    
                       \\ \cline{2-3} 
                       &  \fan{Czysty Kolor}{清一色}{Qīng Yīsè}                        
                       &  Ręka zawierająca kamienie z tylko i wyłącznie jednej z talii numerowanych, nie zawierająca honorów.                    
                       \\ \cline{2-3} 
                       &  \fan{Czysty Potrójny Sekwens}{一色三同顺}{Yīsè Sān Tóngshùn}                        
                       &  Ręka zawierająca trzy identyczne sekwensy w tej samej talii.                     
                       \\ \cline{2-3} 
                       &  \fan{Kolejne Trzy\\Czyste Trójki}{一色三节高}{Yīsè Sān Jiégāo}                        
                       &  Ręka zawierająca trzy trójki (lub \pinyin{gangi}) z 3 kolejnych kamieni w jednej talii.                    
                       \\ \cline{2-3} 
                       &  \fan{Same Górne}{全大}{Quán Dà}                        
                       &  Ręka składająca się tylko i wyłącznie z kamieni o numerach 7, 8 i 9.                    
                       \\ \cline{2-3} 
                       &  \fan{Same Środkowe}{全中}{Quán Zhōng}                        
                       &  Ręka składająca się tylko i wyłącznie z kamieni o numerach 4, 5 i 6.                     
                       \\ \cline{2-3} 
                       &  \fan{Same Dolne}{全小}{Quán Xiǎo}                        
                       &  Ręka składająca się tylko i wyłącznie z kamieni o numerach 1, 2 i 3.                     
                       \\ \hline
\multirow{6}{*}{16}    &  \fan{Czysty Strit}{清龙}{Qīng Lóng}                        
					   &  Ręka zawierająca sekwensy 1-2-3, 4-5-6 i 7-8-9 w jednej talii.                      
					   \\ \cline{2-3} 
                       &  \fan{Trzykolorowe Terminalne\\Sekwensy}{三色双龙会}{Sānsè Shuānglónghuì}                        
                       &  Ręka zawierająca sekwensy 1-2-3 i 7-8-9 w dwóch taliach (łącznie 4 sekwensy) oraz parę 5 w trzeciej.                    
                       \\ \cline{2-3} 
                       &  \fan{}{一色三步高}{Yīsè Sān Bùgāo}                        
                       &  \tabsplit{Ręka zawierająca 3 sekwensy w jednej talii, w których rozpoczynające je (najniższe) kamienie są 3 kolejnymi,}{odległymi od siebie o 1 lub 2.}                     
                       \\ \cline{2-3} 
                       &  \fan{Same Piątki}{全带五}{Quán Dài Wǔ}                        
                       &  Każda grupa (trójka, sekwens lub \pinyin{gang}) lub para  zawiera przynajmniej jeden kamień 5 z dowolnej talii.                     
                       \\ \cline{2-3} 
                       &  \fan{Potrójna Trójka}{三同刻}{Sān Tóngkè}                        
                       &  Ręka zawierająca trójkę (lub \pinyin{gang}) z kamienia o tym samym numerze w każdej z talii.                     
                       \\ \cline{2-3} 
                       &  \fan{Trzy Zakryte Trójki}{三暗刻}{Sān Ànkè}                        
                       &  Ręka zawierająca 3 zakryte trójki (lub \pinyin{gangi}).                     
                       \\ \hline
\multirow{5}{*}{12}    &  \fan{Mniejsze Honory\\i Kamienie Zszywane}{全不靠}{Quán Bù Kào}                        
					   &  \tabsplit{Ręka składająca się z pojedynczych kamieni honorów lub kamieni z sekwencji zszywanych (na przykład 3-6-9 lub 2-5-8) w dowolnej z talii}
					   {(odchodzi od standardowej struktury ręki).}                     
					   \\ \cline{2-3} 
                       &  \fan{Zszywany Strit}{组合龙}{Zǔhé Lóng}                        
                       &  \tabsplit{Ręka, która w miejsce 3 z 4 grup (trójek, sekwensów lub \pinyin{gangów})\\ma 3 różne sekwencje kamieni zszywanych (na przykład 1-4-7, 2-5-8 i 3-6-9)} 
					   {(odchodzi od standardowej struktury ręki).}                         
                       \\ \cline{2-3} 
                       &  \fan{Powyżej Piątki}{大于五}{Dà Yú Wǔ}                        
                       &  Ręka składająca się wyłącznie z kamieni o numerach od 6 do 9 z talii numerowanych.                     
                       \\ \cline{2-3} 
                       &  \fan{Poniżej Piątki}{小于五}{Xiǎo Yú wǔ}                        
                       &  Ręka składająca się wyłącznie z kamieni o numerach od 1 do 4 z talii numerowanych.                     
                       \\ \cline{2-3} 
                       &  \fan{Wielkie Trzy Wiatry}{三风刻}{Sān Fēngkè}                     
                       &  Ręka zawierająca trójki lub \pinyin{gangi} 3 z 4 kamieni wiatrów.                     
                       \\ \hline
\multirow{10}{*}{8}    &  \fan{Mieszany Strit}{花龙}{Huā Lóng}                        
					   &  Strit (sekwensy 1-2-3, 4-5-6 i 7-8-9) składający się z sekwensów w każdej z 3 talii.                    
					   \\ \cline{2-3} 
                       &  \fan{Nieprzewracalne}{推不倒}{Tuī Bù Dǎo}                        
                       &  \tabsplit{Ręka składająca się tylko i wyłącznie z symetrycznych kamieni, czyli 1, 2, 3, 4, 5, 8 i 9 z talii kółek,}
                       	  {2, 4, 5, 6, 8 i 9 z talii bambusów oraz kamienia białego smoka.}                   
                       \\ \cline{2-3} 
                       &  \fan{Mieszany Potrójny Sekwens}{三色三同顺}{Sānsè Sān Tóngshùn}                        
                       &  Trzy sekwensy złożone z kamieni o tych samych numerach w każdej z 3 talii.                    
                       \\ \cline{2-3} 
                       &  \fan{Mieszane Kolejne\\Trzy Trójki}{三色三节高}{Sānsè Sānjié Gāo}                        
                       &  Ręka zawierająca trzy trójki (lub \pinyin{gangi}) z 3 kamieni o kolejnych numerach w każdej z 3 talii.                      
                       \\ \cline{2-3} 
                       &  \fan{Zwycięstwo Bez \pinyin{Fan}}{无番和}{Wúfān Hú}                        
                       &  Ręka, które nie spełnia żadnego innego \pinyin{fan}, czyli jest warta 0 punktów (nie licząc kamieni kwiatów).                     
                       \\ \cline{2-3} 
                       &  \fan{Cudowne Ozdrowienie}{妙手回春}{Miàoshǒuhuíchūn}                        
                       &  Zwycięstwo poprzez dobranie ostatniego kamienia będącego częścią muru.                    
                       \\ \cline{2-3} 
                       &  \fan{Wyławianie Księżyca\\z Dna jeziora}{海底捞月}{Hǎidǐlāoyuè}                        
                       &  \tabsplit{Zwycięstwo poprzez deklarację \pinyin{hu} na ostatnim kamieniu odrzuconym w rozdaniu}{(po którego odrzuceniu nastąpiłby remis, gdyby nie deklaracja). }                   
                       \\ \cline{2-3} 
                       &  \fan{Zwycięstwo\\na Kamieniu Uzupełniającym}{杠上开花}{Gāng Shàng Kāi Huā}                        
                       &  Zwycięstwo na kamieniu uzupełniającym \pinyin{ganga} lub kwiat.                    
                       \\ \cline{2-3} 
                       &  \fan{Okradanie \pinyin{Ganga}}{抢杠和}{Qiǎnggāng Hú}                        
                       &  Zwycięstwo na kamieniu dokładanym do \pinyin{ganga} innego gracza.                     
                       \\ \cline{2-3} 
                       &  \fan{Dwa zakryte \pinyin{Gangi}}{双暗杠}{Shuāng Àngāng}                        
                       &  Ręka zawierająca 2 zakryte \pinyin{gangi}.                     
                       \\ \hline
\multirow{6}{*}{6}     &  \fan{Same Trójki}{碰碰和}{Pèngpèng Hú}                        
					   &  Ręka składająca się z 4 trójek (lub \pinyin{gangów}) i pary.                     
					   \\ \cline{2-3} 
                       &  \fan{Brudny Kolor}{混一色}{Hùn Yīsè}                        
                       &  Ręka składająca się z kamieni należących do jednej talii oraz honorów.                    
                       \\ \cline{2-3} 
                       &  \fan{Mieszane Kolejne Sekwensy}{三色三步高}{Sānsè Sān Bùgāo}                        
                       &  Ręka zawierająca 3 sekwensy o numerach różniących się od siebie kolejno o 1, każdy w innej talii.                     
                       \\ \cline{2-3} 
                       &  \fan{Pięć Typów}{五门齐}{Wǔ Mén Qí}                        
                       &  \tabsplit{Ręka w której każda z grup (trójek, \pinyin{gangów} lub sekwensów) oraz para są złożone z innego typu kamienia,}
                       {czyli jedna musi należeć do każdej z talii (bambusów, kółek oraz liczb), jedna musi się składać ze smoków i jedna z wiatrów.}                   
                       \\ \cline{2-3} 
                       &  \fan{Same Deklaracje}{全求人}{Quán Qiúrén}                        
                       &  \tabsplit{Wszystkie 4 grupy (trójki, \pinyin{gangi} lub sekwensy) na tej ręce muszą być odkryte}{(zadeklarowane z kamieni odrzuconych przez innych graczy).}                   
                       \\ \cline{2-3} 
                       &  \fan{Dwie Trójki Smoków}{双箭刻}{Shuāng Jiànkè}                        
                       &  Ręka zawierająca 2 trójki (lub \pinyin{gangi}) smoków.                     
                       \\ \hline
\multirow{4}{*}{4}     &  \fan{Terminalna Ręka}{全带幺}{Quán Dài Yāo}                        
					   &  Ręka w której każda z grup (trójek, \pinyin{gangów} lub sekwensów) oraz para zawierają co najmniej 1 kamień terminalny.      
					   \\ \cline{2-3} 
                       &  \fan{W Pełni Zakryta Ręka}{不求人}{Bù Qiúrén}                        
                       &  \tabsplit{Zakryta ręka (nie używająca kamieni odrzuconych przez innych graczy) połączona ze zwycięstwem przez \pinyin{zimo}}{(dobranie wygrywającego kamienia z muru).}                     
                       \\ \cline{2-3} 
                       &  \fan{Dwa Odkryte \pinyin{Gangi}}{双明杠}{Shuāng Míng Gāng}                        
                       &  Ręka zawierająca dwa odkryte \pinyin{gangi}.
                       \\ \cline{2-3} 
                       &  \fan{Zwycięstwo\\na Ostatnim Egzemplarzu}{和绝张}{Hú Juézhāng}                        
                       &  \tabsplit{Zwycięstwo na ostatnim (czwartym) egzemplarzu kamienia, w grze. Pozostałe 3 egzemplarze musiały zostać odrzucone}{lub użyte w odkrytych trójkach, sekwensach lub \pinyin{gangach} graczy.}                     
                       \\ \hline
\multirow{10}{*}{2}    &  \fan{Trójka ze Smoków}{箭刻}{Jiànkè}                        
					   &  Ręka zawierająca trójkę z jednego z 3 smoków.                     
					   \\ \cline{2-3} 
                       &  \fan{Wiatr Rundy}{圈风刻}{Quānfēngkè}                        
                       &  Ręka zawierająca trójkę z kamienia wiatru rundy.                     
                       \\ \cline{2-3} 
                       &  \fan{Wiatr Miejsca}{门风刻}{Ménfēngkè}                        
                       &  Ręka zawierająca trójkę z kamienia wiatru miejsca przypisanego do zwycięzcy.                     
                       \\ \cline{2-3} 
                       &  \fan{Zakryta Ręka}{门前清}{Ménqiánqīng}                        
                       &  Ręka nie używająca kamieni odrzuconych przez innych graczy (nie licząc wygrywającego kamienia).                    
                       \\ \cline{2-3} 
                       &  \fan{Płaskie Zwycięstwo}{平和}{Píng Hú} %same sekwensy                        
                       &  Ręka zawierająca 4 sekwensy.                     
                       \\ \cline{2-3} 
                       &  \fan{Cztery Oddzielnie}{四归一}{Sì Guīyī}                        
                       &  Ręka zawierająca 4 egzemplarze tego samego kamienia, ale nie używająca ich jako \pinyin{ganga}.                     
                       \\ \cline{2-3} 
                       &  \fan{Podwójna Trójka}{双同刻}{Shuāng Tóngkè}                        
                       &  Ręka zawierająca 2 trójki kamienia o tym samym numerze w 2 różnych taliach.                     
                       \\ \cline{2-3} 
                       &  \fan{Dwie Zakryte Trójki}{双暗刻}{Shuāng Ànkè}                        
                       &  Ręka zawierająca 2 zakryte trójki (lub \pinyin{gangi}).                     
                       \\ \cline{2-3} 
                       &  \fan{Zakryty \pinyin{Gang}}{暗杠}{Àngāng}                        
                       &  Ręka zawierająca zakrytego \pinyin{ganga}.                     
                       \\ \cline{2-3} 
                       &  \fan{Same Nieterminalne}{断幺}{Duànyāo}                        
                       &  Ręka składająca się tylko i wyłącznie z kamieni o numerach od 2 do 8 z talii numerowanych.                     
                       \\ \hline
\multirow{13}{*}{1}    &  \fan{Podwójny Czysty Sekwens}{一般高}{Yī Bān Gāo}                        
					   &  Ręka zawierająca 2 identyczne sekwensy w tej samej talii.                     
					   \\ \cline{2-3} 
                       &  \fan{Mieszany Podwójny Sekwens}{喜相逢}{Xǐxiāngféng}                       
                       &  Ręka zawierająca 2 sekwensy używające kamieni o tych samych numerach w 2 różnych taliach.                     
                       \\ \cline{2-3} 
                       &  \fan{Sześć Kolejnych}{连六}{Lián Liù}                        
                       &  Ręka zawierająca 2 sekwensy złożone z 6 kolejnych kamieni w tej samej talii.                    
                       \\ \cline{2-3} 
                       &  \fan{Dwa Terminalne Sekwensy}{老少副}{Lǎoshàofù}                        
                       &  Ręka zawierajaca sekwensy 1-2-3 i 7-8-9 w tej samej talii.                    
                       \\ \cline{2-3} 
                       &  \fan{Trójka\\Terminalnych lub Honorów}{Yāojiǔkè}{}                        
                       &  Ręka zawierająca trójkę (lub \pinyin{ganga}) z honorów lub kamieni terminalnych.                     
                       \\ \cline{2-3} 
                       &  \fan{Odkryty \pinyin{Gang}}{明杠}{Míng Gāng}                        
                       &  Ręka zawierajaca odkrytego \pinyin{ganga}.                     
                       \\ \cline{2-3} 
                       &  \fan{Brak Jednej Talii}{缺一门}{Quē Yī Mén}                        
                       &  Ręka, w której nie zostały użyte kamienie z jednej z 3 talii numerowanych.                     
                       \\ \cline{2-3} 
                       &  \fan{Bez Honorów}{无字}{Wú Zì}                        
                       &  Ręka nie zawierająca honorów.                     
                       \\ \cline{2-3} 
                       &  \fan{Skrajne Czekanie}{边张}{Biānzhāng}                        
                       &  Ręka, w której wygrywający kamień to 3 do sekwensu 1-2-3 lub 7 do sekwensu 7-8-9.                     
                       \\ \cline{2-3} 
                       &  \fan{Zamknięte Czekanie}{坎张}{Kǎnzhāng}                        
                       &  Ręka, w której wygrywający kamień to środkowy kamień sekwensu (na przykład 5 dla sekwensu 4-5-6).                    
                       \\ \cline{2-3} 
                       &  \fan{Czekanie do Pary}{单调将}{Dāndiàojiāng}                        
                       &  Ręka, w której wygrywającym kamieniem może być tylko jeden brakujący kamień do pary.                     
                       \\ \cline{2-3} 
                       &  \fan{Zwycięstwo z Muru}{自摸}{Zìmō}                        
                       &  Zwycięstwo poprzez dobranie wygrywającego kamienia z muru.                     
                       \\ \cline{2-3} 
                       &  \fan{Kwiaty}{花牌}{Huāpái}                        
                       &  \tabsplit{Każdy zadeklarowany kamień kwiat jest \pinyin{fan} wartym 1 punkt (maksymalnie 8).}
                          {Punkty z tego \pinyin{fan} nie są liczone do
                          minimalnych 8 punktów wygrywającej ręki.\label{tab:fan} } \\ \hline
%\caption{(\pinyin{Shìjiè Májiàng Zǔzhī} 2014)} 
%
\end{longtable}
\normalsize