\onecolumn
\section*{Bibliografia}
Carlisle, Rodney P. 2009. ,,Mahjong'' W: \textit{Encyclopedia of Play in Today's
Society}, 371-372.
Sage Publications.

Culin, Stewart 1924. ,,The Game of Ma-Jong'' W: \textit{Brooklyn Museum
Quarterly}, Volume XI, 153-168.


Greenfield, Mary C. 2010. „The Game of One Hundred Intelligences: ,,Mahjong,
Materials, and the Marketing of the Asian Exotic in the 1920s''. W: Carl Abott i
David A. Johnson (ed.), \textit{Pacific Historical Review}, Vol. 79. 329-359.
California: University of California Press.

Harr, Lew Lyle 2008. \textit{Pung Chow -- The Game of a Hundred Intelligences.
Also known as Mah-Diao, Mah-Jong, Mah-Cheuk, Mah-Juck and Pe-Ling}. Project
Gutenberg

金学诗 (Jīn Xuéshī) 1783. \textit{牧猪闲话 (Mùzhū Xiánhuà; Plotki o Świniopasach)}.
昭代丛书 (\pinyin{Zhāodài Cóngshū})

Kanazawa, Masahiro 2010. 《2002 世界麻雀選手権大会について》 (\pinyin{Sekaimājansenshuken
taikai ni tsuite} -- \textit{O Światowych Mistrzostwach Świata w Madżongu
2002}).
\\http://www1.tcue.ac.jp/home1/takamatsu/100124/100124-002.pdf [dostęp
2016-04-25]

梓沫 (Zǐ Mò) 2015.《裁判解读国标麻将:八番起‘胡’降低运气成分》(\pinyin{Cáipàn jiědú guóbiāo májiàng: Bā
fān qǐ ‘hé’ jiàngdī yùnqì chéngfèn} -- \textit{Interpretacja sędziego krajowego madżonga:
hu przy 8 fan zmniejsza szczęście}) % tutaj jest o nowych zasadach z 1998
\\http://sports.sina.com.cn/go/2015-10-25/doc-ifxizwsi5588602.shtml [dostęp 2016-04-25]

Mah Jong Museum. \textit{Mah Jong Historical Documents}
\\http://www.mahjongmuseum.com/history.htm [dostęp
2016-03-30]


日本健康麻将協会公式サイト (\pinyin{Nihon kenkō asa shō kyōkai kōshiki saito}) 2002.
2002 年世界麻雀選手権大会個人部門成績 (\pinyin{2002 nen Sekaimājansenshuken Taikai Kojin
Bumon Seiseki} -- Wyniki Indywidualne Światowych Zawodów Madżonga
2002) \\http://www.kenko-mahjong.com/topics/sekaikojin.htm [dostęp 2016-04-25]

Sloper, Tom; inni autorzy 2006. \textit{The CC Theory}
\\http://www.sloperama.com/cctheory/detail.html [dostęp
2016-03-30]

Stanwick, Michael 2004. ,,Mahjong(g) Before Mahjong(g): Part 1''. W:
\textit{The Playing-card}, Vol. 32, No. 4, 153-162

Stanwick, Michael 2004. ,,Mahjong(g) Before Mahjong(g): Part 2''. W:
\textit{The Playing-card}, Vol. 32, No. 5, 206-215
% \\http://www.themahjongtileset.co.uk/tile-set-history/mahjongg-before-mahjongg-part-1/ 
% \\http://www.themahjongtileset.co.uk/tile-set-history/mahjongg-before-mahjongg-part-2/
% [dostęp
% 2016-03-30]

Stanwick, Michael 2006. ,,Mahjong(g) Before and After Mahjong(g): Part 1''. W:
\textit{The Playing-card}, Vol. 34, No. 4, 259-268

Stanwick, Michael 2006. ,,Mahjong(g) Before and After Mahjong(g): Part 2''. W:
\textit{The Playing-card}, Vol. 35, No. 1, 27-39

% Stanwick, Michael 2006. \textit{Mahjong(g) Before And After Mahjong(g)}
% %\\http://www.themahjongtileset.co.uk/tile-set-history/mahjongg-before-and-after-mahjongg-part-1/
%  \\http://www.themahjongtileset.co.uk/tile-set-history/mahjongg-before-and-after-mahjongg-part-2/
% [dostęp
% 2016-03-30]

Stanwick, Michael; Xu, Hongbing 2012. ,,From Cards to Tiles: The Origin of
Mahjong(g)’s Earliest Suit Names''. W:\textit{The Playing-card}, Vol. 41, No. 1,
52–67.

Stanwick, Michael; Xu, Hongbing. \textit{Mahjong terms 1780 – 1920}
\\http://www.themahjongtileset.co.uk/mahjong-terms-1780-1920/ [dostęp
2016-03-30]

%Stanwick, Michael; Xu, Hongbing. \textit{From Cards to Tiles: The Origin of
% Mahjong(g)’s Earliest Suit Names}
%\\http://www.themahjongtileset.co.uk/tile-set-history/earliest-suit-names/
Wikipedia 2013, \textit{George B. Glover}
\\https://en.wikipedia.org/wiki/George\_B.\_Glover [dostęp 2016-04-22]

Wikipedia 2016, \textit{Stewart Culin}
\\https://en.wikipedia.org/wiki/Stewart\_Culin [dostęp 2016-04-22]

World Heritage Encyclopedia 2014. \textit{Mahjong}
\\http://www.worldheritage.org/article/WHEBN0000019496/Mahjong [dostęp
2016-03-31]

口装巴士 (\pinyin{Kǒuzhuāng Bāshì}) 2014. 《乐麻将延伸之世界麻将比赛介绍》 (\pinyin{Huānlè májiàng
yánshēn zhī shìjiè májiàng bǐsài jièshào} - \textit{Popularyzacja madżonga
poprzez wprowadzenie światowych zawodów sportowych})
\\http://m.ptbus.com/huanlemajiang/221298/ [dostęp 2016-04-25]

世界麻将组织 (\pinyin{Shìjiè Májiàng Zǔzhī}) 2006. 《麻将竞赛规则》 (\pinyin{Májiàng Jìngsài
Guīzé} -- \textit{Światowe Zasady Madżonga})
\\http://www.chinamajiang.com/adobe\%20reader/z20140402.pdf [dostęp 2016-01-04]





