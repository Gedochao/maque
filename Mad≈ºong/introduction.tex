\introduction
% %%%%%%%%%%%%%%%%%%%%%%%%%%%%%%%%%%%%%%%%%%%%%%%%%%%%%%%%%%%%%%%%%%%%%%%%
% WSTĘP MERYTORYCZNY
% %%%%%%%%%%%%%%%%%%%%%%%%%%%%%%%%%%%%%%%%%%%%%%%%%%%%%%%%%%%%%%%%%%%%%%%%
% %%%%%%%%%%%%%%%%%%%%%%%%%%%%%%%%%%%%%%%%%%%
% DEFINICJE
% %%%%%%%%%%%%%%%%%%%%%%%%%%%%%%%%%%%%%%%%%%%%%
\section{Czym jest madżong?}
% %%%%%%%%%%%%%%%%%%%%%%%%%%%%%%%%%%%%%%%%%%%
% DEFINICJA Z ENCYKLOPEDII
% %%%%%%%%%%%%%%%%%%%%%%%%%%%%%%%%%%%%%%%%%%%%%
\subsection{Definicja encyklopedyczna}
Madżong (麻将 \pinyin{májiàng} lub 麻雀 \pinyin{máquè}) to znana na całym świecie
gra dla czterech graczy pochodząca z Chin. Do gry używa się zestawu 144 kamieni
z wygrawerowanymi (niekiedy malowanymi) z jednej strony znakami chińskimi i
innymi symbolami. W trakcie rozgrywki gracze po kolei dobierają i odrzucają
kamienie w celu uzbierania jednego z ustalonych układów. Choć największą
popularność osiągnęła w Azji, ma również duże grupy odbiorców na pozostałych
kontynentach.
Istnieje kilkadziesiąt wariantów jej zasad, w większości powiązanych z
konkretnym regionem lub państwem. Rozbieżności pomiędzy zasadami poszczególnych
wariantów są w niektórych przypadkach bardzo duże, niektóre uwzględniają też grę
dla mniejszej liczby graczy (madżong dla trzech osób jest szczególnie popularny
w Japonii oraz Korei) (World Heritage Encyclopedia 2014).
% %%%%%%%%%%%%%%%%%%%%%%%%%%%%%%%%%%%%%%%%%%%
% NAPOTKANE NIEŚCISŁOŚCI
% %%%%%%%%%%%%%%%%%%%%%%%%%%%%%%%%%%%%%%%%%%%%%
\subsection{Zastrzeżenia do definicji}
\label{zastrzezenia}
Autor pracy uważa, że powyższa definicja, choć poprawna, może nie być
wystarczająco precyzyjna na potrzeby niniejszej pracy.

Jedną z podstawowych nieścisłości jest określenie kamienia będącego częścią
zestawu do gry.
Niezależnie od tego, czy mówimy o kamieniach, płytkach domino czy też o kartach
używanych do bardzo różnego typu gier, określane są one w języku chińskim
wspólnym mianem \pinyin{pai} (牌 \pinyin{pái}). W wyniku tego, w wielu źródłach
autor nie może być pewien, czy opisywana gra wymagała kamieni, kart, czy jeszcze
innego typu elementów.

Innym problemem okazuje się być skład zestawu do gry. Wiele spośród
historycznych zestawów z terenu Chin różni się pomiędzy sobą typem i liczbą
figur w stopniu zdecydowanie większym, niż pozwalają na to uznane na terenie
Państwa Środka warianty zasad gry.

Kolejne wątpliwości wzbudza sposób gry. Niektóre ze źródeł sugerują, że
najstarsze zestawy do gry zbliżone w swoim składzie i wyglądzie do współczesnych
były wykorzystywane do gier innych niż madżong (czyli o zasadach
nieprzypominających żadnego ze współcześnie akceptowanych wariantów).

W związku z wymienionymi problemami autor decyduje się używać bardziej
precyzyjnej definicji gry, %(opisanej dalej na stronie \pageref{definicja}), 
która będzie obowiązywać w niniejszej pracy.

% %%%%%%%%%%%%%%%%%%%%%%%%%%%%%%%%%%%%%%%%%%%
% DEFINICJA PRZYJĘTA W PRACY
% %%%%%%%%%%%%%%%%%%%%%%%%%%%%%%%%%%%%%%%%%%%%%
% %%%%%%%%%%%%%%%%%%%%%%%%%%%%%%%%%%%%%%%%%%%
% DEFINICJA OBOWIĄZUJĄCA W PRACY
% %%%%%%%%%%%%%%%%%%%%%%%%%%%%%%%%%%%%%%%%%%%%%
%\mahjongdef % oddzielna sekcja w spisie treści (,,drugi wstęp'')
\def \traditionalstonesfootnote {\getrefnumber{definicja_tradycyjne}} %zmienna do referencji
% przypisu przy kamieniach zapisanych w znakach tradycyjnych
\subsection{Definicja przyjęta na potrzeby niniejszej pracy}
\label{definicja}
Madżong to gra spełniająca %definicję opisaną w sekcji 1.1.1 niniejszej pracy
% oraz
następujące warunki:

% %%%%%%%%%%%%%%%% WARUNEK A
\begin{enumerate}[label={\alph*)}] \item Gra wykorzystuje zestaw składający się
z kamieni, które kształtem przypominają płytki domina, jednakże są nieco grubsze
i krótsze. Dopuszczalna jest także możliwość wykorzystywania kart zamiast
kamieni.
% %%%%%%%%%%%%%%%% WARUNEK B
\item Kamienie (lub karty) mają grawerowane lub malowane oznaczenia,
przyporządkowujące je do odpowiednich grup figur lub talii. Zestaw uwzględnia
następujące talie i figury\footnote{Skład zestawu do gry został opisany w tej
sekcji uwzględniając wszystkie dopuszczalne jego możliwości.
Precyzyjny opis współczesnego zestawu do gry w madżonga zgodny z
międzynarodowymi zasadami turniejowymi znajduje się  na stronie
\pageref{guobiao_zestaw}.}:
	\begin{itemize}
	  \item 3 talie (kółka (筒 \pinyin{tǒng} lub 饼 \pinyin{bǐng}), bambusy (索
	  \pinyin{suǒ}) oraz liczby (万 \pinyin{wàn})) z kamieniami o
wartościach od 1 do 9; dopuszczalna jest także talia rang zamiast talii liczby
(oznaczana przez znaki 卐 (\pinyin{wàn}, dla kamienia o wartości 1) oraz 品 
(\pinyin{pǐn}, dla kamieni o wartościach od 2 do 9)); 
\item 4 wiatry \label{wiatry}
(wschód(東\footnote{\label{definicja_tradycyjne}Znaki wyjątkowo zapisane w formie
tradycyjnej, jako że w takiej występują na kamieniach. Dla znaku 東  forma
uproszczona to 东, dla 將 jest to 将, dla 發 -- 发, dla 龍 -- 龙, a dla 鳳 -- 凤.}
\pinyin{dōng}), południe (南 \pinyin{nán}), zachód (西 \pinyin{xī}) i północ (北
\pinyin{běi}));  dopuszczalny jest także wariant, w którym zamiast 4 wiatrów
występuje 4 możnych (diuk (公 \pinyin{gōng}),  markiz (侯 \pinyin{hóu}),  generał
(將\footnotemark[\traditionalstonesfootnote] \pinyin{jiàng}) oraz minister (相
\pinyin{xiàng}));
	  \item 3 smoki (biały (白 \pinyin{bái}), zielony
	  (發\footnotemark[\traditionalstonesfootnote] \pinyin{fā}) i czerwony (中
	  \pinyin{zhòng})); dopuszczalny jest także wariant, w którym zamiast smoka
	  zielonego jest smok bez przydzielonej mu barwy\footnote{,,3 Smoki'' w
	  kolorach białym, zielonym i czerwonym to zasadniczo termin zachodni, nie
	  chiński.
	  Chińczycy używają określenia ,,3 \pinyin{yuan}'' (三元牌 \pinyin{sān yuán
	  pái}). We wczesnych zestawach do gry w madżonga występował jednakże
	  kamień oznaczany znakiem smoka --
	  龍\footnotemark[\traditionalstonesfootnote].}
	  (龍\footnotemark[\traditionalstonesfootnote] \pinyin{lóng}), a zamiast smoka
	  czerwonego -- feniks
	  (鳳\footnotemark[\traditionalstonesfootnote]\pinyin{fèng});
	  \item opcjonalnie kwiaty i/lub pory roku w różnej ilości;
	  \item opcjonalnie dżoker lub inne kamienie specjalne.
	\end{itemize}
%%%%%%%%%%%%%%%%%
% WARUNEK C
\item Kamienie o odpowiednich oznaczeniach występują w obrębie jednego zestawu
do gry w następujących ilościach:
	\begin{itemize}
	  \item 4 egzemplarze każdego z kamieni o numerach od 1 do 9 w każdej z 3 talii
	  (łącznie 108 kamieni);
	  \item 4 egzemplarze każdego z kamieni wiatrów (łącznie 16 kamieni);
	  \item 4 egzemplarze kamieni każdego z 3 smoków (łącznie 12 kamieni);
	  \item łączna liczba kamieni kwiatów i pór roku może się wahać od 0 do ponad
	  20;
	  \item łączna liczba kamieni specjalnych oraz dżokerów może się wahać od
	0 do ponad 20.
	\end{itemize} 
%%%%%%%%%%%%%%%%%
% WARUNEK D
\item Zasady gry uwzględniają następujące punkty:
	\begin{itemize}
	  \item gra polega na kompletowaniu przez graczy określonego przez zasady
	  układu kamieni zwanego dalej ,,ręką''\footnote{\label{reka}W niektórych
	  kontekstach termin ,,ręka'' może oznaczać również pełne jedno rozdanie w czasie gry (盘
	  \pinyin{pán}) (Shìjiè Májiàng Zǔzhī 2014). Aby uniknąć niejednoznaczności,
	  na potrzeby niniejszej pracy termin ten nie będzie używany w tym znaczeniu.}
	  (手牌 \pinyin{shǒupái});
	  \item gracz, który skompletuje gotową rękę jako pierwszy, wygrywa rozdanie;
	  \item choć liczba kamieni tworzących rękę może ulec zmianie w zależności od
	  wariantu gry, musi się ona składać z pewnej liczby ,,grup'' (pomniejszych
	  układów składających się z od 3 kamieni\footnote{Spotykane są również grupy
	  składające się z 4 lub 5 kamieni, jednakże wymagają one specjalnych
	  deklaracji w czasie gry oraz dobrania dodatkowych kamieni na wymianę. W
	  rezultacie, potencjalny czwarty lub piąty kamień w grupie nie wpływa na
	  budowę ręki, lecz na późniejsza punktację, stąd są one dalej traktowane
	  jako kamienie bonusowe.}) oraz dokładnie jednej pary (2 kamieni tego samego
	  typu), zgodnie z poniższym wzorem:
	  \begin{equation*}
	  n = 3g + 2 + x
	  \end{equation*}
	  gdzie n to liczba kamieni tworzących rękę, g to liczba grup, a x to liczba
	  kamieni bonusowych\footnote{Kamienie bonusowe nie zawsze są traktowane jako
	  część ręki gracza, jednakże są z nią ściśle powiązane, w związku z czym
	  autor pracy uznał za stosowne uwzględnić je we wzorze.} (kwiatów lub
	  rozszerzeń grup do 4 lub 5 kamieni);
	  \item gra jest skierowana do 4 graczy (opcjonalnie zasady mogą
	  zezwalać na grę dla 3 graczy).
	\end{itemize} 
\end{enumerate}
(Sloper 2006; Stanwick \& Xu; Shìjiè Májiàng Zǔzhī 2006)

% %%%%%%%%%%%%%%%%%%%%%%%%%%%%%%%%%%%%%%%%%%%
% CEL PRACY I STRESZCZENIE ROZDZIAŁÓW
% %%%%%%%%%%%%%%%%%%%%%%%%%%%%%%%%%%%%%%%%%%%%%
\section{Cel i treść pracy}
Celem niniejszej pracy jest opisanie historii gry w madżonga, okoliczności jego
powstania i przemian, a także zestawienie ze sobą najważniejszych odmian jego
zasad.

W rozdziale \ref{historia} opisane zostało pochodzenie oraz historia rozwoju gry
w madżonga, a także jej rola w kulturze popularnej.

W rozdziale \ref{guobiao} znajduje się szczegółowe wyjaśnienie reguł gry zgodnie
z wariantem międzynarodowych zasad turniejowych.

W rozdziale \ref{zestawienie_zasad} znajduje się skrócony opis innych spośród
najważniejszych odmian madżonga, ich porównanie oraz tło kulturowe.

Na stronie \pageref{suplement} znajduje się suplement do pracy, który zawiera
tabelę układów punktowanych zgodną z międzynarodowymi zasadami turniejowymi
(opisanymi w rozdziale \ref{guobiao}).

% %%%%%%%%%%%%%%%%%%%%%%%%%%%%%%%%%%%%%%%%%%%%%%%%%%%%%%%%%%%%%%%%%%%%%%%%
% WSTĘP TECHNICZNY
% %%%%%%%%%%%%%%%%%%%%%%%%%%%%%%%%%%%%%%%%%%%%%%%%%%%%%%%%%%%%%%%%%%%%%%%%
\section{Przyjęte formy zapisu oraz transkrypcje}
W niniejszej pracy terminy i tytuły w języku chińskim
\pinyin{putonghua}\footnote{\pinyin{Putonghua} (普通话 \pinyin{pǔtōnghuà}) --
oficjalny standard mówionego języka chińskiego; język urzedowy Chińskiej Republiki
Ludowej, Republiki Chińskiej oraz jeden z czterech języków urzędowych
Singapuru.} zapisane są w znakach uproszczonych oraz transkrypcji fonetycznej
\pinyin{Hanyu Pinyin} (汉语拼音 \pinyin{Hànyǔ Pīnyīn}), natomiast te w języku
kantońskim -- w znakach tradycyjnych i transkrypcji fonetycznej
\jyutping{Jyutping} (粵拼 \jyutping{Jyut6ping3}). Wyjątkami są niektóre nazwy
kamieni do gry w madżonga, które zapisane są w znakach tradycyjnych, jako że
kamieni nie oznacza się znakami uproszczonymi. Zapis tych nazw w znakach
uproszczonych umieszczony jest w adnotacjach.

Terminy i nazwy japońskie zapisane są pismem japońskim
(\romaji{hiraganą}\footnote{\romaji{Hiragana} (平仮名 lub ひらがな \romaji{hiragana})
-- jeden z dwóch rodzajów japońskiego pisma sylabicznego. Używany jest głównie do
zapisu partykuł, końcówek gramatycznych oraz słów, które nie posiadają
własnego zapisu \romaji{kanji} lub których zapis \romaji{kanji} nie jest
popularny.}, \romaji{katakaną}\footnote{\romaji{Katakana} (片仮名 lub かたかな
\romaji{katakana}) -- jeden z dwóch rodzajów japońskiego pisma sylabicznego.
Używany jest głównie do zapisu słów pochodzenia obcego.} i przy użyciu
\romaji{kanji}\footnote{\romaji{Kanji} (漢字 \romaji{kanji}) -- znaki pochodzenia
chińskiego używane w piśmie japońskim.})  i w
transkrypcji fonetycznej Hepburna (ヘボン式ローマ字 \romaji{Hebon-shiki}
\romaji{Rōmaji}).
