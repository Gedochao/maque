\introduction
% %%%%%%%%%%%%%%%%%%%%%%%%%%%%%%%%%%%%%%%%%%%%%%%%%%%%%%%%%%%%%%%%%%%%%%%%
% WSTĘP MERYTORYCZNY
% %%%%%%%%%%%%%%%%%%%%%%%%%%%%%%%%%%%%%%%%%%%%%%%%%%%%%%%%%%%%%%%%%%%%%%%%
% %%%%%%%%%%%%%%%%%%%%%%%%%%%%%%%%%%%%%%%%%%%
% DEFINICJE
% %%%%%%%%%%%%%%%%%%%%%%%%%%%%%%%%%%%%%%%%%%%%%
\section{Czym jest madżong?}
% %%%%%%%%%%%%%%%%%%%%%%%%%%%%%%%%%%%%%%%%%%%
% DEFINICJA Z ENCYKLOPEDII
% %%%%%%%%%%%%%%%%%%%%%%%%%%%%%%%%%%%%%%%%%%%%%
\subsection{Powszechnie przyjęta definicja}
Madżong (麻将 \pinyin{májiàng} lub 麻雀 \pinyin{máquè}) to znana na całym świecie
gra dla czterech graczy pochodząca z Chin. Do gry używa się zestawu 144 kamieni
z wygrawerowanymi (niekiedy malowanymi) z jednej strony znakami chińskimi i
innymi symbolami. W trakcie rozgrywki gracze po kolei dobierają i odrzucają
kamienie w celu uzbierania jednego z ustalonych układów. Choć największą
popularność osiągnęła w Azji, ma również duże grupy odbiorców na pozostałych
kontynentach.
Istnieje kilkadziesiąt wariantów jej zasad, w większości powiązanych z
konkretnym regionem lub państwem. Rozbieżności pomiędzy zasadami poszczególnych
wariantów są w niektórych przypadkach bardzo duże, niektóre uwzględniają też grę
dla mniejszej liczby graczy (madżong dla trzech osób jest szczególnie popularny
w Japonii oraz Korei) (World Heritage Encyclopedia 2014).
% %%%%%%%%%%%%%%%%%%%%%%%%%%%%%%%%%%%%%%%%%%%
% NAPOTKANE NIEŚCISŁOŚCI
% %%%%%%%%%%%%%%%%%%%%%%%%%%%%%%%%%%%%%%%%%%%%%
\subsection{Zastrzeżenia do definicji}
\label{zastrzezenia}
Autor pracy uważa, że powyższa definicja, choć poprawna, może nie być
wystarczająco precyzyjna na potrzeby niniejszej pracy.

Jedną z podstawowych nieścisłości jest określenie kamienia będącego częścią
zestawu do gry.
Niezależnie od tego, czy mówimy o kamieniach, płytkach domino czy też o kartach
używanych do bardzo różnego typu gier, określane są one w języku chińskim
wspólnym mianem \pinyin{pai} (牌 \pinyin{pái}). W wyniku tego, w wielu źródłach
autor nie może być pewien, czy opisywana gra wymagała kamieni, kart, czy jeszcze
innego typu elementów.

Innym problemem okazuje się być skład zestawu do gry. Wiele spośród
historycznych zestawów z terenu Chin różni się pomiędzy sobą typem i liczbą
figur w stopniu zdecydowanie większym, niż pozwalają na to uznane na terenie
Państwa Środka warianty zasad gry.

Kolejne wątpliwości wzbudza sposób gry. Niektóre ze źródeł sugerują, że
najstarsze zestawy do gry zbliżone w swoim składzie i wyglądzie do współczesnych
były wykorzystywane do gier innych niż madżong (czyli o zasadach
nieprzypominających żadnego ze współcześnie akceptowanych wariantów).

W związku z wymienionymi problemami autor decyduje się używać bardziej
precyzyjnej definicji gry (opisanej dalej na stronie \pageref{definicja}), która
będzie obowiązywać w niniejszej pracy.
% %%%%%%%%%%%%%%%%%%%%%%%%%%%%%%%%%%%%%%%%%%%
% CEL PRACY I STRESZCZENIE ROZDZIAŁÓW
% %%%%%%%%%%%%%%%%%%%%%%%%%%%%%%%%%%%%%%%%%%%%%
\section{Cel i treść pracy}
Celem niniejszej pracy jest zestawienie ze sobą najważniejszych odmian madżonga.

W rozdziale \ref{historia} opisane zostało pochodzenie oraz historia rozwoju gry
w madżonga.

W rozdziale \ref{guobiao} znajduje się szczegółowe wyjaśnienie reguł gry zgodnie
z wariantem międzynarodowych zasad turniejowych.

W rozdziale \ref{zestawienie_zasad} znajduje się skrócony opis innych spośród
najważniejszych odmian madżonga oraz ich porównanie.
% %%%%%%%%%%%%%%%%%%%%%%%%%%%%%%%%%%%%%%%%%%%%%%%%%%%%%%%%%%%%%%%%%%%%%%%%
% WSTĘP TECHNICZNY
% %%%%%%%%%%%%%%%%%%%%%%%%%%%%%%%%%%%%%%%%%%%%%%%%%%%%%%%%%%%%%%%%%%%%%%%%
\section{Przyjęte formy zapisu oraz transkrypcje}
W niniejszej pracy terminy i tytuły w języku chińskim
\pinyin{putonghua}\footnote{\pinyin{Putonghua} (普通话 pǔtōnghuà) -- oficjalny
standard mówionego języka chińskiego; język urzedowy Chińskiej Republiki
Ludowej, Republiki Chińskiej oraz jeden z czterech języków urzędowych
Singapuru.} zapisane są w znakach uproszczonych oraz transkrypcji fonetycznej
\pinyin{Hanyu Pinyin} (汉语拼音 \pinyin{Hànyǔ Pīnyīn}), natomiast te w języku
kantońskim -- w znakach tradycyjnych i transkrypcji fonetycznej
\jyutping{Jyutping} (粵拼 \jyutping{Jyut6ping3}). Wyjątkami są niektóre nazwy
kamieni do gry w madżonga, które zapisane są w znakach tradycyjnych, jako że
kamieni nie oznacza się znakami uproszczonymi. Zapis tych nazw w znakach
uproszczonych umieszczony jest w adnotacjach.

Terminy i nazwy japońskie zapisane są pismem japońskim
(\romaji{hiraganą}\footnote{\romaji{Hiragana} (平仮名 lub ひらがな \romaji{hiragana})
-- jeden z dwóch rodzajów japońskiego pisma sylabicznego. Używany jest głównie do
zapisu partykuł, końcówek gramatycznych oraz słów, które nie posiadają
własnego zapisu \romaji{kanji} lub których zapis \romaji{kanji} nie jest
popularny.}, \romaji{katakaną}\footnote{\romaji{Katakana} (片仮名 lub かたかな
\romaji{katakana}) -- jeden z dwóch rodzajów japońskiego pisma sylabicznego.
Używany jest głównie do zapisu słów pochodzenia obcego.} i przy użyciu
\romaji{kanji}\footnote{\romaji{Kanji} (漢字 \romaji{kanji}) -- znaki pochodzenia
chińskiego używane w piśmie japońskim.})  i w
transkrypcji fonetycznej Hepburna (ヘボン式ローマ字 \romaji{Hebon-shiki Rōmaji}).
